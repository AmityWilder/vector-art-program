\documentclass[12pt]{article}

\usepackage{graphicx}
\usepackage{amssymb,amsthm,amsmath}
\usepackage[margin=1in]{geometry}
\usepackage{parskip}
\usepackage{fancyhdr}
\usepackage{tikz}
\usepackage{txfonts}
\usepackage{esvect}
\pagestyle{fancy}

\newtheorem{theorem}{Theorem}
\newtheorem{lemma}[theorem]{Lemma}
\newtheorem{proposition}{Proposition}
\newtheorem{corollary}[theorem]{Corollary}

\newcommand{\Z}{\mathbb{Z}}
\newcommand{\Q}{\mathbb{Q}}
\newcommand{\N}{\mathbb{N}}
\newcommand{\R}{\mathbb{R}}
\newcommand{\C}{\mathbb{C}}
\newcommand{\zmod}[1]{\mathbb{Z}_{#1}}
\renewcommand{\mod}[3]{{#1} \equiv {#2} \pmod {#3}}

\lhead{Amy Wilder}
\rhead{Minimizing B\'ezier Curve Points\dots}

\begin{document}

\begin{center}
    {\Large Minimizing Cubic B\'ezier Curve Control Points From Ongoing Signal While Retaining High Accuracy}
\end{center}

Given an ongoing, two-dimensional signal, we will create a cubic B\'ezier curve to approximate that signal while using the smallest reasonable number of control points.

Our original signal will be \(s(t) = \langle s_x(t),s_y(t) \rangle\), and the ``pressure'' of that signal will be \(s_p(t)\).
Let the change in pressure, \(\frac{\Delta s_p}{\Delta t}(t)\), be known. Let the array into which we place our control points be \(P(t) = [\dots]\).

Stipulations:
\begin{itemize}
    \item If \(s_p(t) = 0\) and \(\frac{\Delta s_p}{\Delta t}(t) = 0\), then \(P\) is immutable.
    \item If \(s_p(t) \ne 0\) or \(\frac{\Delta s_p}{\Delta t}(t) \ne 0\), then \(P\) is mutable.
    \item If \(\frac{\Delta s_p}{\Delta t}(t) \ne 0\), then \({P(t)}_{|P(t)|}\) must be exactly \(s(t)\), either through mutation or insertion.
\end{itemize}

\section*{Case 1: Point}

The signal is a ``point'' if \(\frac{\Delta s_p}{\Delta t}(t - \Delta t) = 1\) and \(\frac{\Delta s_p}{\Delta t}(t) = -1\).
For this case, we must insert the current signal.
\begin{align*}
    P(t) = P(t - \Delta t) \cup \begin{cases}
        \left[s(t)\right] & \text{if } \frac{\Delta s_p}{\Delta t}(t - \Delta t) = 1 \bigwedge \frac{\Delta s_p}{\Delta t}(t) = -1 \\
        \varnothing & \text{otherwise}
    \end{cases}
\end{align*}

\section*{Case 2: Point}

Assume \(|P(t)| \ge 2\).
Let \(\vec{p_{-1}} = {P(t)_{|P(t)|}}\) and \(\vec{p_0} = s(t)\).

\section*{Case 3: Straight Edge}

Assume \(|P(t)| \ge 2\).
Let \(\vec{p_{-1}} = {P(t)_{|P(t)|}}\) and \(\vec{p_0} = s(t)\).
The signal is a ``straight edge'' if

\section*{Case 4: Hard Corner}

\section*{Case 5: Wide Angle}

\section*{Case 6: Smooth Turn}

\section*{Case 7: Inflection}

\end{document}
