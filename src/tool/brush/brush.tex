\documentclass[12pt]{article}

\usepackage{graphicx}
\usepackage{amssymb,amsthm,amsmath}
\usepackage[margin=1in]{geometry}
\usepackage{parskip}
\usepackage{fancyhdr}
\usepackage{tikz}
\pagestyle{fancy}

\newtheorem{theorem}{Theorem}
\newtheorem{lemma}[theorem]{Lemma}
\newtheorem{proposition}{Proposition}
\newtheorem{corollary}[theorem]{Corollary}

\newcommand{\Z}{\mathbb{Z}}
\newcommand{\Q}{\mathbb{Q}}
\newcommand{\N}{\mathbb{N}}
\newcommand{\R}{\mathbb{R}}
\newcommand{\C}{\mathbb{C}}
\newcommand{\zmod}[1]{\mathbb{Z}_{#1}}
\renewcommand{\mod}[3]{{#1} \equiv {#2} \pmod {#3}}

\lhead{Amy Wilder}
\rhead{Minimizing B\'ezier Curve Points\dots}

\begin{document}

\begin{center}
    {\Large Minimizing Cubic B\'ezier Curve Control Points From Ongoing Signal While Retaining High Accuracy}
\end{center}

Given an ongoing, two-dimensional signal, we will create a cubic B\'ezier curve to approximate that signal while using the smallest reasonable number of control points.

Our original signal will be \(s(t) = \langle s_x(t),s_y(t) \rangle\), and the pressure of that signal will be \(s_p(t)\).
Assume \(\frac{\Delta s_p}{\Delta t}(t)\) is known.

Requirements:
\begin{itemize}
    \item We must never add a point while \(s_p(t) = 0\).
    \item We must always add a point while \(\frac{\Delta s_p}{\Delta t}(t) \ne 0\).
    \item If \(s_p(t) \ne 0\), whether we add a point will depend on the criteria we are defining.
\end{itemize}

\section*{Case 1: Point}

Assume \(\frac{\Delta s_p}{\Delta t}(t - \Delta t) = 1\) and \(\frac{\Delta s_p}{\Delta t}(t) = -1\).

\section*{Case 2: Edge}

\section*{Case 3: Corner}

\section*{Case 4: Angle}

\section*{Case 5: Turn}

\section*{Case 6: Inflection}

\begin{center}
    \begin{tikzpicture}
        \draw (0,0) circle[radius=2pt];
        \draw (3,0) circle[radius=2pt];
    \end{tikzpicture}
\end{center}

\end{document}
